\subsection{Multilingual Experiments}
\label{sec:multilang}
\noindent Following Christodoulopoulos et
al. \shortcite{christodoulopoulos-goldwater-steedman:2011:EMNLP}, we extend our
experiments to 8 languages from MULTEXT-East (Bulgarian, Czech, English,
Estonian, Hungarian, Romanian, Slovene and Serbian) \cite{citeulike:5820223}
and 10 languages from the CoNLL-X shared task (Bulgarian, Czech, Danish, Dutch,
German, Portuguese, Slovene, Spanish, Swedish and Turkish)
\cite{Buchholz:2006:CST:1596276.1596305}.  

To sample substitutes, we trained language models of Bulgarian, Czech,
Estonian, Romanian, Danish, German, Dutch, Portuguese, Spanish, Swedish and
Turkish with their corresponding TenTen corpora \cite{jakubivcek2013tenten},
and Hungarian, Slovene and Serbian with their corresponding Wikipedia dump
files\footnote{Latest Wikipedia dump files are freely available at
  \url{http://dumps.wikimedia.org/} and the text in the dump files can be
  extracted using WP2TXT (\url{http://wp2txt.rubyforge.org/})}.  Serbian shares
  a common basis with Crotian and Bosnian therefore we trained 3 different
  language models using Wikipedia dump files of Serbian together with these two
  languages and measured the perplexities on the MULTEXT-East Serbian corpus.
  We chose the Croatian language model since it achieved the lowest perplexity
  score and unknown word ratio on MULTEXT-East Serbian corpus.  We use ukWaC
  corpora to train English language models. 

We used the default settings in Section~\ref{sec:expset} and incorporated only
the orthographic features\footnote{All corpora (except German, Spanish and
Swedish) label the punctuation marks with the same gold-tag therefore we add an
extra {\em punctuation} feature for those languages.}.  Extracting
unsupervised morphological features for languages with different
characteristics would be of great value, but it is beyond the scope of this
paper.  For each language the number of induced clusters is set to the number
of tags in the gold-set.  To perform meaningful comparisons with the previous
work we train and evaluate our models on the training section of
MULTEXT-East\footnote{Languages of MULTEXT-East corpora do not tag the
punctuations, thus we add an extra tag for punctuations to the tag-set of these
languages.} and CONLL-X languages \cite{Lee:2010:STU:1870658.1870741}.

Table~\ref{tab:multiresults} presents the performance of our instance based
model on 19 corpora in 15 languages together with the corresponding best
published results from
$^\diamond$\protect\cite{yatbaz-sert-yuret:2012:EMNLP-CoNLL},
$^\ddagger$\protect\cite{blunsom-cohn:2011:ACL-HLT2011},
$^\star$\protect\cite{christodoulopoulos-goldwater-steedman:2011:EMNLP} and
$^\dagger$\protect\cite{Clark:2003:CDM:1067807.1067817}.  All of the
state-of-the-art systems in Table~\ref{tab:multiresults} are word-based and
incorporate morphological features. 
%%Multiple language section
\begin{table}[t]
\centering  \small
%% Table notation info
%% PYP is \ddagger
%% BMMM is \star
%% Clark is \dagger
%% kmeans is ^k
%% Yatbaz \diamond  
%% 
  \begin{tabular}{l|l|l|l|l|}
    \hline
    & Language   & Tags & \specialcell{Best\\ Published \\ \mto\ \ \vm\ } &
    \specialcell{Instance\\Based\\\mto\ / \vm\ }\\\hline 
    \multirow{1}{*}{\begin{sideways}\textbf{WSJ}\end{sideways}} 
    & English     & 45   & .802 / {\bf .721} $^\diamond$        &.795 / .691\\ 
    & & & &\\ \hline
    \multirow{8}{*}{\begin{sideways}\textbf{MULTEXT-East}\end{sideways}}
    & Bulgarian   & 12+1 & .665 / {\bf .556}$^\star$                  & .664 / .513\\ 
    & Czech       & 12+1 & .642 / {\bf.539}$^\star$                   & {\bf .705} / .511\\ 
    & English     & 12+1 & .733 / .633$^\star$                        & {\bf .835} / {\bf .661}\\ 
    & Estonian    & 11+1 & .644 / {\bf.533}$^\star$                   & .643 / .457\\ 
    & Hungarian   & 12+1 & {\bf .682} / {\bf.548}$^\star$             & .647/ .459\\
    & Romanian    & 14+1 & .611 / .523$^\star$                        & {\bf .660} / .528\\
    & Slovene     & 12+1 & {\bf .679} / {\bf.567}$^\star$             & .667 / .451\\
    & Serbian     & 12+1 & {\bf .641} / {\bf.510}$\dagger$            & .594/ .402\\
    \hline % Conll06 data
    \multirow{10}{*}{\begin{sideways}\textbf{CoNLL-X Shared Task}\end{sideways}}
    & Bulgarian   & 54   & .704 / {\bf.596}$\dagger$                  & {\bf .751} / .583\\
    & Czech       & 12   & .701$^\ddagger$ / .484$^\star$             & .701 / .486\\
    & Danish      & 25   & .761$^\ddagger$ / .591$^\star$             & .761 / .584\\
    & Dutch       & 13   & .711$^\ddagger$ / {\bf .547}$^\star$       & .712 / .537\\
    & German      & 54   & .744$^\star$ / {\bf .630}$\dagger$         & .749 / .618\\
    & Portuguese  & 22   & .785$^\ddagger$ / {\bf .639}$^\star$       & .782 / .607\\
    & Slovene     & 29   & .642$^\star$ / {\bf.539}$\dagger$          & .638 / .469\\
    & Spanish     & 47   & {\bf.788}$^\ddagger$ / {\bf .632}$^\star$  & .753/ .602\\
    & Swedish     & 41   & .682 / {\bf.589}$\dagger$                  & .681 / .546\\
    & Turkish     & 30   & .628 / .408$^\star$                        & .637 / .401\\ 
    \hline
  \end{tabular}
  \caption{The \mto\ and \vm\ scores on 19 corpora in 15 languages together
  with number of induced clusters.  Statistically significant results shown
  in bold ($p < 0.05$).
}  
  \label{tab:multiresults}
\end{table}


Our \mto\ results are lower than the best systems on all of data-sets that use
language models trained on the Wikipedia corpora.  ukWaC and TenTen corpora are
cleaner and tokenized better compared to the Wikipedia corpora.  These corpora
also have larger vocabulary sizes and lower out-of-vocabulary rates.  Thus
language models trained on these corpora have much lower perplexities and
generate better substitutes than the Wikipedia based models.  Our model has
lower \vm\ scores in spite of good \mto\ scores on 14 corpora which is
discussed in Section~\ref{sec:discuss}.

Among the languages for which clean language model corpora were available, our
model performs comparable to or significantly better than the best systems on
most languages.  We show significant improvements on MULTEXT-East Czech,
Romanian, and CoNLL-X Bulgarian.  Our model achieves the state-of-the-art \mto\
on MULTEXT-East English and scores comparable \mto\ with the best model on WSJ.
Our model shows comparable results on MULTEXT-East Bulgarian and Estonian, and
CoNLL-X Czech, Danish, Dutch, German, Portuguese, Swedish and Turkish in terms
of the \mto\ score.  One reason for comparably low \mto\ on Spanish might be
the absence of morphological features.

% TODO: describe the table.

%% Morphological features of each language are extracted by the
%% method described in Section~\ref{sec:feat}.  The details of the
%% language model training and feature extraction are detailed in
%% Appendix~D.
%% We ignore these results
%% %%Multiple language section
\begin{table*}[ht]
%% Table notation info
%% PYP is \ddagger
%% BMMM is \start
%% Clark is \dagger
%% kmeans is ^k
%% 
%  \small
  \centering
  \caption{The \mto\ and \vm\ scores of X+Y token clustering S-CODE on
    19 corpora in 15 languages together with the number of unique words and
    tags in gold--set which equals to number of induced clusters in
    all languages.  MULTEXT-East corpora do not tag the punctuation
    marks, thus we add an extra tag for punctuation and represent it
    with ``+1''.}
%  \begin{tabular}{|@{ }l@{ }|@{ }l@{ }|@{ }l@{ }|@{ }l@{ }|@{ }l@{ }|@{ }l@{ }|}
  \begin{tabular}{|l|l|l|l|l|l|}
        \hline
        & Language   & Types    & Tags & S-CODE X & S-CODE X+Y \\ \hline % Multext data
        \multirow{1}{*}{\begin{sideways}\textbf{WSJ}\end{sideways}} 
        & English    & 49,190  & 45 & .7667 / .6819 &.7030 / .6006\\
        & & & & &\\ \hline
        \multirow{8}{*}{\begin{sideways}\textbf{MULTEXT-East}\end{sideways}}
        & Bulgarian  & 16,352  & 12+1 & .6927 / .5341 & .6551 / .4644\\
        & Czech      & 19,115  & 12+1 & .7025 / .5020 & .6145 / .4031\\
        & English    & 9,773   & 12+1 & .8239 / .6631 & .7697 / .5977\\
        & Estonian   & 17,845  & 11+1 & .6612 / .4469 & .5934 / .3509\\
        & Hungarian  & 20,321  & 12+1 & .6900 / .4972 & .6235 / .4152\\
        & Romanian   & 15,189  & 14+1 & .6412 / .5004 & .5931 / .4185\\
        & Slovene    & 17,871  & 12+1 & .6914 / .4951 & .6466 / .4119\\
        & Serbian    & 18,095  & 12+1 & .6244 / .4473 & .5622 / .3466\\
        \hline % Conll06 data
        \multirow{10}{*}{\begin{sideways}\textbf{CoNLL-X Shared Task}\end{sideways}}
        & Bulgarian  & 32,439  & 54 & .7328 / .5781 & .6615 / .4821\\
        & Czech      & 130,208 & 12 & .6739 / .4838 & .6082 / .3566\\
        & Danish     & 18,356  & 25 & .7236 / .5583 & .6210 / .4351\\
        & Dutch      & 28,393  & 13 & .6957 / .5331 & .6168 / .4000\\
        & German     & 72,326  & 54 & .7669 / .6308 & .6484 / .5083\\
        & Portuguese & 28,931  & 22 & .7479 / .5798 & .6734 / .4859\\
        & Slovene    & 7,128   & 29 & .6513 / .4957 & .5918 / .4164\\
        & Spanish    & 16,458  & 47 & .7479 / .6086 & .6739 / .5142\\
        & Swedish    & 20,057  & 41 & .6962 / .5674 & .5779 / .4385\\
        & Turkish    & 17,563  & 30 & .6239 / .3823 & .5910 / .3309\\ \hline
    \end{tabular}
  \label{tab:multiresults}
\end{table*}

%% \subsubsection{Results}
%% \label{sec:multires}
%% For each language we report results of three models that cluster: (1)
%% word embeddings ({\em CLU-W}), (2) word embeddings with orthographic
%% features ({\em CLU-W+O}) and (3) word embeddings with both orthographic
%% and morphological features ({\em CLU-W+O+M}).  
%% As a baseline model we chose the syntagmatic bigram version of S-CODE
%% described in Section~\ref{sec:pvss} which is a very strong baseline
%% compared to the ones used in
%% \cite{christodoulopoulos-goldwater-steedman:2011:EMNLP}.
%% Table~\ref{tab:multiresults} summarizes the \mto\ and \vm\ scores of
%% our models together with the syntagmatic bigram baseline and the best
%% published accuracies on each language corpus.
%% {\em CLU-W} significantly outperforms the syntagmatic bigram baseline
%% in both \mto\ and \vm\ scores on 14 languages.  {\em CLU-W+O+M} has
%% the state-of-the-art \mto\ and \vm\ accuracy on the PTB.  {\em
%%   CLU-W+O} and {\em CLU-W+O+M} achieve the highest \mto\ scores on all
%% languages of MULTEXT-East corpora while scoring the highest \vm\
%% accuracies on English and Romanian.  On the CoNLL-X languages our
%% models perform better than the best published \mto\ or \vm\ accuracies
%% on 10 languages.

